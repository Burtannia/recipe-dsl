\documentclass[11pt]{article}

\usepackage[hyphens]{url}

\usepackage{hyperref}
\hypersetup
{
    breaklinks = true,
    allcolors = black,
    linktoc = all
}

\usepackage{mathtools}
\usepackage{amssymb}
\usepackage{stmaryrd}
\usepackage{graphicx}

\usepackage{multicol}
\setlength{\columnseprule}{0.4pt}

\usepackage{listings}

\usepackage[english]{babel}
\usepackage[utf8]{inputenc}
\usepackage{fancyhdr}
\pagestyle{fancy}
\fancyhf{}
\rhead{James Burton}
\lhead{ID: 4251529}
\fancyfoot[C]{\thepage}

\usepackage{titling}
\title
{ 
    \vspace{10em}
    G53IDS - Final Report \\
    \hfill \break
    \large Embedded Domain Specific Language for \\
    Describing Recipes in Haskell
}

\setlength{\droptitle}{-10em}

\author{James Burton - 4251529 - psyjb6}

\begin{document}
    \maketitle
    \newpage

    \tableofcontents
    \newpage

    \section{Abstract and Acknowledgements}
    \newpage

    \section{Introduction and Motivation}
    \subsection{Overview}
    \subsection{Cooking with Computers}
    \subsection{DSLs and Haskell}

    \section{Development Process}
    \subsection{Project Management}
    
    \section{Combinators}
        \subsection{Initial Definitions}
        \subsection{Sequencing Problem}
        \subsection{Conditionals}
        \subsection{Transactions}
        \subsection{Moving to a Tree of Actions}
        \subsection{Final Definitions}
        \subsection{Custom Combinators}

    \section{Deriving Equality}
        \subsection{Topological Sorting}
        \subsection{Quickspec}
    
    \section{Modelling a Kitchen}
        \subsection{Stations}
        \subsection{Scheduling}
            \subsubsection{Linear Programming}
            \subsubsection{Implementing Scheduler}
        
    \section{Recipe Properties}
        \subsection{Folding Over Recipes}
        \subsection{Time and Cost}
        \subsection{Flavours}
        \subsection{Generating New Recipes}

    \section{Evaluation and Testing}
        \subsection{Test Recipes}
        \subsection{QuickCheck}

    \section{Summary and Reflections}
        \subsection{Project Management}
        \subsection{Contributions}
        \subsection{Reflections}

    \section{Related Work}

    \newpage

    \begin{thebibliography}{8}

        \bibitem{robot}
        The Guardian. 2015. \textit{Future of food: how we cook}.
        \url{https://www.theguardian.com/technology/2015/sep/13/future-of-food-how-we-cook}

        \bibitem{hudak}
        Paul Hudak. Domain Specific Languages. Department of Computer
        Science, Yale University, December 15, 1997.

        \bibitem{snoyman}
        Michael Snoynman. O'Reilly Webcast: Designing Domain Specific
        Languages with Haskell. January 4, 2013.
        \url{https://www.youtube.com/watch?v=8k_SU1t50M8}

        \bibitem{contracts}
        Simon Peyton Jones, Microsoft Research, Cambridge.
        Jean-Marc Eber, LexiFi Technologies, Paris. Julian Seward,
        University of Glasgow. Composing contracts: an adventure in
        financial engineering. August 17, 2000.

        \bibitem{pretty}
        John Hughes. The Design of a Pretty-printing Library.
        Chalmers Teniska Hogskola, Goteborg, Sweden. 1995.

        \bibitem{contracts-pp}
        Simon Peyton Jones, Microsoft Research, Cambridge.
        Jean-Marc Eber, LexiFi Technologies, Paris. Julian Seward,
        University of Glasgow. Composing contracts: an adventure in
        financial engineering (PowerPoint Slides). August 17, 2000.
        \url{https://www.microsoft.com/en-us/research/publication/composing-contracts-an-adventure-in-financial-engineering/}

        \bibitem{core}
        Simon Peyton Jones. Into the Core - Squeezing Haskell into
        Nine Constructors. September 14, 2016.
        \url{https://www.youtube.com/watch?v=uR_VzYxvbxg}

        \bibitem{hutton-fold}
        Graham Hutton. Fold and Unfold for Program Semantics. Department of
        Computer Science, University of Nottingham. September 1998.

    \end{thebibliography}   

    \newpage

    \appendix

\end{document}