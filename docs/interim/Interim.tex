\documentclass[11pt]{article}

\usepackage[hyphens]{url}

\usepackage{hyperref}
\hypersetup
{
    breaklinks = true,
    allcolors = black,
    linktoc = all
}

\usepackage{graphicx}

\usepackage{listings}

\usepackage[english]{babel}
\usepackage[utf8]{inputenc}
\usepackage{fancyhdr}
\pagestyle{fancy}
\fancyhf{}
\rhead{James Burton}
\lhead{ID: 4251529}
\fancyfoot[C]{\thepage}

\usepackage{titling}
\title
{ 
    G53IDS - Interim Report \\
    \hfill \break
    \large Embedded Domain Specific Language for \\
    Describing Recipes in Haskell
}

\setlength{\droptitle}{-10em}

\author{James Burton - 4251529 - psyjb6}

\begin{document}
    \maketitle
    \newpage

    \tableofcontents
    \newpage

    \section{Introduction}
    Consider the following recipe to make a cup of tea:

    \begin{tt}
    \begin{lstlisting}
    - Boil some water
    - Pour over a teabag
    - Wait for 3 minutes
    - Remove the teabag
    - Add milk (optional)
    \end{lstlisting}
    \end{tt}

    This is a very simple but useful recipe that many people
    will perform, in some cases, many times a day over their lives.
    What we can realise by looking at this recipe is that it actually
    consists of many smaller recipes, such as boiling water and
    combining tea with milk, performed in a certain order. This
    raises the question, to what extent does the order matter and
    to what extent can we rearrange things in order to make the recipe
    more efficient? No doubt you have done this, maybe subconsciously,
    while cooking at home. Furthermore which steps can be done
    concurrently in the event that multiple people are cooking e.g.
    in a professional kitchen with a full brigade?

    Perhaps closer to computer science, we could also ask, how could
    we instruct a robot to do this? After doing some research on
    robotic chefs it appears that not a huge number exist.
    There is one home cooking robot \cite{robot} which uses motion
    capture in order to learn recipes. In my opinion this is rather
    restrictive. It presumes that the human performs the recipe in
    the optimal manner and it would be very difficult to model
    a brigade system in this way. In reality there is a limited set
    of fundamental actions that one becomes able to perform when
    learning to cook. Recipes can then be performed using a sequence
    of these actions. Representing recipes like this would allow us
    to take a robot programmed to perform each of the fundamental
    actions and tell it how to cook literally anything.

    So we've established that if we can structure recipes in a more
    formal way then we will have a great amount of freedom in terms
    of how we process them whether it be optimisation for a human
    chef or full on automation. My contributions / planned contributions
    to this are the following:

    \begin{itemize}
        \item Define a set of combinators as an EDSL in Haskell and show that
        they can be used to describe a wide variety of recipes (Section 2).

        \item The combinators describe a recipe but we then need to know
        how to execute a recipe as a sequence of the fundamental actions
        mentioned above. Fortunately, as described above, recipes are just
        a combination of simpler recipes meaning that we can recursively
        define the actions necessary to perform complex recipes as long
        as we have manually defined which action are required for each
        combinator (Section 3).

        \item We now have a way to describe recipes and a way to show the
        sequence of actions to complete a recipe but now we need to apply
        them to something. Using the operational semantics of recipes
        we can optimise and schedule recipes for a given kitchen system.
        We can then express this in several ways including printing steps,
        drawing the cooking process as a graph or simulating the recipe
        within the given kitchen setup (Section 4).

        \item It may also be useful to provide an intermediate language
        between informal English and our combinators, some sort of markup
        language, that can then be parsed in.
    \end{itemize}

    \section{Describing Recipes}

    \section{Executing a Recipe}

    \section{Implementation}

    \section{Progress}
    \subsection{Project Management}
    \subsection{Contributions and Reflections}

    \section{Related Work}

    \newpage
    \begin{thebibliography}{0}
        \bibitem{robot}
        The Guardian. 2015. \textit{Future of food: how we cook}.
        \url{https://www.theguardian.com/technology/2015/sep/13/future-of-food-how-we-cook}
    \end{thebibliography}   
     
\end{document}